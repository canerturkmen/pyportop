% Generated by Sphinx.
\def\sphinxdocclass{report}
\documentclass[letterpaper,10pt,english]{sphinxmanual}
\usepackage[utf8]{inputenc}
\DeclareUnicodeCharacter{00A0}{\nobreakspace}
\usepackage{cmap}
\usepackage[T1]{fontenc}
\usepackage{babel}
\usepackage{times}
\usepackage[Bjarne]{fncychap}
\usepackage{longtable}
\usepackage{sphinx}
\usepackage{multirow}


\title{Portfolio Optimization Module (POM) Documentation}
\date{May 24, 2014}
\release{v0.1}
\author{AC Turkmen}
\newcommand{\sphinxlogo}{}
\renewcommand{\releasename}{Release}
\makeindex

\makeatletter
\def\PYG@reset{\let\PYG@it=\relax \let\PYG@bf=\relax%
    \let\PYG@ul=\relax \let\PYG@tc=\relax%
    \let\PYG@bc=\relax \let\PYG@ff=\relax}
\def\PYG@tok#1{\csname PYG@tok@#1\endcsname}
\def\PYG@toks#1+{\ifx\relax#1\empty\else%
    \PYG@tok{#1}\expandafter\PYG@toks\fi}
\def\PYG@do#1{\PYG@bc{\PYG@tc{\PYG@ul{%
    \PYG@it{\PYG@bf{\PYG@ff{#1}}}}}}}
\def\PYG#1#2{\PYG@reset\PYG@toks#1+\relax+\PYG@do{#2}}

\expandafter\def\csname PYG@tok@gd\endcsname{\def\PYG@tc##1{\textcolor[rgb]{0.63,0.00,0.00}{##1}}}
\expandafter\def\csname PYG@tok@gu\endcsname{\let\PYG@bf=\textbf\def\PYG@tc##1{\textcolor[rgb]{0.50,0.00,0.50}{##1}}}
\expandafter\def\csname PYG@tok@gt\endcsname{\def\PYG@tc##1{\textcolor[rgb]{0.00,0.27,0.87}{##1}}}
\expandafter\def\csname PYG@tok@gs\endcsname{\let\PYG@bf=\textbf}
\expandafter\def\csname PYG@tok@gr\endcsname{\def\PYG@tc##1{\textcolor[rgb]{1.00,0.00,0.00}{##1}}}
\expandafter\def\csname PYG@tok@cm\endcsname{\let\PYG@it=\textit\def\PYG@tc##1{\textcolor[rgb]{0.25,0.50,0.56}{##1}}}
\expandafter\def\csname PYG@tok@vg\endcsname{\def\PYG@tc##1{\textcolor[rgb]{0.73,0.38,0.84}{##1}}}
\expandafter\def\csname PYG@tok@m\endcsname{\def\PYG@tc##1{\textcolor[rgb]{0.13,0.50,0.31}{##1}}}
\expandafter\def\csname PYG@tok@mh\endcsname{\def\PYG@tc##1{\textcolor[rgb]{0.13,0.50,0.31}{##1}}}
\expandafter\def\csname PYG@tok@cs\endcsname{\def\PYG@tc##1{\textcolor[rgb]{0.25,0.50,0.56}{##1}}\def\PYG@bc##1{\setlength{\fboxsep}{0pt}\colorbox[rgb]{1.00,0.94,0.94}{\strut ##1}}}
\expandafter\def\csname PYG@tok@ge\endcsname{\let\PYG@it=\textit}
\expandafter\def\csname PYG@tok@vc\endcsname{\def\PYG@tc##1{\textcolor[rgb]{0.73,0.38,0.84}{##1}}}
\expandafter\def\csname PYG@tok@il\endcsname{\def\PYG@tc##1{\textcolor[rgb]{0.13,0.50,0.31}{##1}}}
\expandafter\def\csname PYG@tok@go\endcsname{\def\PYG@tc##1{\textcolor[rgb]{0.20,0.20,0.20}{##1}}}
\expandafter\def\csname PYG@tok@cp\endcsname{\def\PYG@tc##1{\textcolor[rgb]{0.00,0.44,0.13}{##1}}}
\expandafter\def\csname PYG@tok@gi\endcsname{\def\PYG@tc##1{\textcolor[rgb]{0.00,0.63,0.00}{##1}}}
\expandafter\def\csname PYG@tok@gh\endcsname{\let\PYG@bf=\textbf\def\PYG@tc##1{\textcolor[rgb]{0.00,0.00,0.50}{##1}}}
\expandafter\def\csname PYG@tok@ni\endcsname{\let\PYG@bf=\textbf\def\PYG@tc##1{\textcolor[rgb]{0.84,0.33,0.22}{##1}}}
\expandafter\def\csname PYG@tok@nl\endcsname{\let\PYG@bf=\textbf\def\PYG@tc##1{\textcolor[rgb]{0.00,0.13,0.44}{##1}}}
\expandafter\def\csname PYG@tok@nn\endcsname{\let\PYG@bf=\textbf\def\PYG@tc##1{\textcolor[rgb]{0.05,0.52,0.71}{##1}}}
\expandafter\def\csname PYG@tok@no\endcsname{\def\PYG@tc##1{\textcolor[rgb]{0.38,0.68,0.84}{##1}}}
\expandafter\def\csname PYG@tok@na\endcsname{\def\PYG@tc##1{\textcolor[rgb]{0.25,0.44,0.63}{##1}}}
\expandafter\def\csname PYG@tok@nb\endcsname{\def\PYG@tc##1{\textcolor[rgb]{0.00,0.44,0.13}{##1}}}
\expandafter\def\csname PYG@tok@nc\endcsname{\let\PYG@bf=\textbf\def\PYG@tc##1{\textcolor[rgb]{0.05,0.52,0.71}{##1}}}
\expandafter\def\csname PYG@tok@nd\endcsname{\let\PYG@bf=\textbf\def\PYG@tc##1{\textcolor[rgb]{0.33,0.33,0.33}{##1}}}
\expandafter\def\csname PYG@tok@ne\endcsname{\def\PYG@tc##1{\textcolor[rgb]{0.00,0.44,0.13}{##1}}}
\expandafter\def\csname PYG@tok@nf\endcsname{\def\PYG@tc##1{\textcolor[rgb]{0.02,0.16,0.49}{##1}}}
\expandafter\def\csname PYG@tok@si\endcsname{\let\PYG@it=\textit\def\PYG@tc##1{\textcolor[rgb]{0.44,0.63,0.82}{##1}}}
\expandafter\def\csname PYG@tok@s2\endcsname{\def\PYG@tc##1{\textcolor[rgb]{0.25,0.44,0.63}{##1}}}
\expandafter\def\csname PYG@tok@vi\endcsname{\def\PYG@tc##1{\textcolor[rgb]{0.73,0.38,0.84}{##1}}}
\expandafter\def\csname PYG@tok@nt\endcsname{\let\PYG@bf=\textbf\def\PYG@tc##1{\textcolor[rgb]{0.02,0.16,0.45}{##1}}}
\expandafter\def\csname PYG@tok@nv\endcsname{\def\PYG@tc##1{\textcolor[rgb]{0.73,0.38,0.84}{##1}}}
\expandafter\def\csname PYG@tok@s1\endcsname{\def\PYG@tc##1{\textcolor[rgb]{0.25,0.44,0.63}{##1}}}
\expandafter\def\csname PYG@tok@gp\endcsname{\let\PYG@bf=\textbf\def\PYG@tc##1{\textcolor[rgb]{0.78,0.36,0.04}{##1}}}
\expandafter\def\csname PYG@tok@sh\endcsname{\def\PYG@tc##1{\textcolor[rgb]{0.25,0.44,0.63}{##1}}}
\expandafter\def\csname PYG@tok@ow\endcsname{\let\PYG@bf=\textbf\def\PYG@tc##1{\textcolor[rgb]{0.00,0.44,0.13}{##1}}}
\expandafter\def\csname PYG@tok@sx\endcsname{\def\PYG@tc##1{\textcolor[rgb]{0.78,0.36,0.04}{##1}}}
\expandafter\def\csname PYG@tok@bp\endcsname{\def\PYG@tc##1{\textcolor[rgb]{0.00,0.44,0.13}{##1}}}
\expandafter\def\csname PYG@tok@c1\endcsname{\let\PYG@it=\textit\def\PYG@tc##1{\textcolor[rgb]{0.25,0.50,0.56}{##1}}}
\expandafter\def\csname PYG@tok@kc\endcsname{\let\PYG@bf=\textbf\def\PYG@tc##1{\textcolor[rgb]{0.00,0.44,0.13}{##1}}}
\expandafter\def\csname PYG@tok@c\endcsname{\let\PYG@it=\textit\def\PYG@tc##1{\textcolor[rgb]{0.25,0.50,0.56}{##1}}}
\expandafter\def\csname PYG@tok@mf\endcsname{\def\PYG@tc##1{\textcolor[rgb]{0.13,0.50,0.31}{##1}}}
\expandafter\def\csname PYG@tok@err\endcsname{\def\PYG@bc##1{\setlength{\fboxsep}{0pt}\fcolorbox[rgb]{1.00,0.00,0.00}{1,1,1}{\strut ##1}}}
\expandafter\def\csname PYG@tok@kd\endcsname{\let\PYG@bf=\textbf\def\PYG@tc##1{\textcolor[rgb]{0.00,0.44,0.13}{##1}}}
\expandafter\def\csname PYG@tok@ss\endcsname{\def\PYG@tc##1{\textcolor[rgb]{0.32,0.47,0.09}{##1}}}
\expandafter\def\csname PYG@tok@sr\endcsname{\def\PYG@tc##1{\textcolor[rgb]{0.14,0.33,0.53}{##1}}}
\expandafter\def\csname PYG@tok@mo\endcsname{\def\PYG@tc##1{\textcolor[rgb]{0.13,0.50,0.31}{##1}}}
\expandafter\def\csname PYG@tok@mi\endcsname{\def\PYG@tc##1{\textcolor[rgb]{0.13,0.50,0.31}{##1}}}
\expandafter\def\csname PYG@tok@kn\endcsname{\let\PYG@bf=\textbf\def\PYG@tc##1{\textcolor[rgb]{0.00,0.44,0.13}{##1}}}
\expandafter\def\csname PYG@tok@o\endcsname{\def\PYG@tc##1{\textcolor[rgb]{0.40,0.40,0.40}{##1}}}
\expandafter\def\csname PYG@tok@kr\endcsname{\let\PYG@bf=\textbf\def\PYG@tc##1{\textcolor[rgb]{0.00,0.44,0.13}{##1}}}
\expandafter\def\csname PYG@tok@s\endcsname{\def\PYG@tc##1{\textcolor[rgb]{0.25,0.44,0.63}{##1}}}
\expandafter\def\csname PYG@tok@kp\endcsname{\def\PYG@tc##1{\textcolor[rgb]{0.00,0.44,0.13}{##1}}}
\expandafter\def\csname PYG@tok@w\endcsname{\def\PYG@tc##1{\textcolor[rgb]{0.73,0.73,0.73}{##1}}}
\expandafter\def\csname PYG@tok@kt\endcsname{\def\PYG@tc##1{\textcolor[rgb]{0.56,0.13,0.00}{##1}}}
\expandafter\def\csname PYG@tok@sc\endcsname{\def\PYG@tc##1{\textcolor[rgb]{0.25,0.44,0.63}{##1}}}
\expandafter\def\csname PYG@tok@sb\endcsname{\def\PYG@tc##1{\textcolor[rgb]{0.25,0.44,0.63}{##1}}}
\expandafter\def\csname PYG@tok@k\endcsname{\let\PYG@bf=\textbf\def\PYG@tc##1{\textcolor[rgb]{0.00,0.44,0.13}{##1}}}
\expandafter\def\csname PYG@tok@se\endcsname{\let\PYG@bf=\textbf\def\PYG@tc##1{\textcolor[rgb]{0.25,0.44,0.63}{##1}}}
\expandafter\def\csname PYG@tok@sd\endcsname{\let\PYG@it=\textit\def\PYG@tc##1{\textcolor[rgb]{0.25,0.44,0.63}{##1}}}

\def\PYGZbs{\char`\\}
\def\PYGZus{\char`\_}
\def\PYGZob{\char`\{}
\def\PYGZcb{\char`\}}
\def\PYGZca{\char`\^}
\def\PYGZam{\char`\&}
\def\PYGZlt{\char`\<}
\def\PYGZgt{\char`\>}
\def\PYGZsh{\char`\#}
\def\PYGZpc{\char`\%}
\def\PYGZdl{\char`\$}
\def\PYGZhy{\char`\-}
\def\PYGZsq{\char`\'}
\def\PYGZdq{\char`\"}
\def\PYGZti{\char`\~}
% for compatibility with earlier versions
\def\PYGZat{@}
\def\PYGZlb{[}
\def\PYGZrb{]}
\makeatother

\begin{document}

\maketitle
\tableofcontents
\phantomsection\label{index::doc}


Contents:
\begin{quote}\begin{description}
\item[{maxdepth}] \leavevmode
4

\end{description}\end{quote}
\begin{quote}\begin{description}
\item[{toctree}] \leavevmode
DIRNAME

\end{description}\end{quote}
\phantomsection\label{index:module-model.models}\index{model.models (module)}\index{IBar (class in model.models)}

\begin{fulllineitems}
\phantomsection\label{index:model.models.IBar}\pysiglinewithargsret{\strong{class }\code{model.models.}\bfcode{IBar}}{\emph{*args}, \emph{**kwargs}}{}
Model encapsulates a single timespan price data of an instrument in the POM system

\end{fulllineitems}

\index{Instrument (class in model.models)}

\begin{fulllineitems}
\phantomsection\label{index:model.models.Instrument}\pysiglinewithargsret{\strong{class }\code{model.models.}\bfcode{Instrument}}{\emph{*args}, \emph{**kwargs}}{}
Model encapsulates a single trading instrument (a.k.a security) in the POM system,
in most of the anticipated cases a Forex trading pair
\index{get\_bar\_collection() (model.models.Instrument method)}

\begin{fulllineitems}
\phantomsection\label{index:model.models.Instrument.get_bar_collection}\pysiglinewithargsret{\bfcode{get\_bar\_collection}}{\emph{period}}{}
Returns the set of bar objects that belong to a certain instrument, for a given periodicity
\begin{quote}\begin{description}
\item[{Parameters}] \leavevmode
\textbf{period} ({\hyperref[index:model.models.Period]{\emph{model.models.Period}}}) -- the period for the data, if no data for this periodicity is available, no data will be returned

\item[{Returns}] \leavevmode
an ordered list of IBar objects

\item[{Return type}] \leavevmode
list

\end{description}\end{quote}

\end{fulllineitems}

\index{get\_bar\_collection\_timeframe() (model.models.Instrument method)}

\begin{fulllineitems}
\phantomsection\label{index:model.models.Instrument.get_bar_collection_timeframe}\pysiglinewithargsret{\bfcode{get\_bar\_collection\_timeframe}}{\emph{period}, \emph{start}, \emph{end}}{}
Returns the set of bar objects that belong to a certain instrument, for a given periodicity, and between
a given start and end time
\begin{quote}\begin{description}
\item[{Parameters}] \leavevmode\begin{itemize}
\item {} 
\textbf{period} ({\hyperref[index:model.models.Period]{\emph{model.models.Period}}}) -- the periodicity for the date, if no data for this periodicity is available, no data will be returned

\item {} 
\textbf{start} (\emph{datetime.datetime}) -- the start datetime for the data

\item {} 
\textbf{end} (\emph{datetime.datetime}) -- the end datetime for the data

\end{itemize}

\item[{Returns}] \leavevmode
an ordered list of IBar objects

\item[{Return type}] \leavevmode
list

\end{description}\end{quote}

\end{fulllineitems}

\index{get\_ccy1() (model.models.Instrument method)}

\begin{fulllineitems}
\phantomsection\label{index:model.models.Instrument.get_ccy1}\pysiglinewithargsret{\bfcode{get\_ccy1}}{}{}
Get the 1st currency (CCY1) in the pair (i.e. base currency)
\begin{quote}\begin{description}
\item[{Returns}] \leavevmode
ISO 4217 format currency pair name of CCY1

\end{description}\end{quote}

\end{fulllineitems}

\index{get\_ccy2() (model.models.Instrument method)}

\begin{fulllineitems}
\phantomsection\label{index:model.models.Instrument.get_ccy2}\pysiglinewithargsret{\bfcode{get\_ccy2}}{}{}
Get the 2nd currency (CCY2) in the pair,
\begin{quote}\begin{description}
\item[{Returns}] \leavevmode
ISO 4217 format currency pair name of CCY2

\end{description}\end{quote}

\end{fulllineitems}


\end{fulllineitems}

\index{Period (class in model.models)}

\begin{fulllineitems}
\phantomsection\label{index:model.models.Period}\pysiglinewithargsret{\strong{class }\code{model.models.}\bfcode{Period}}{\emph{*args}, \emph{**kwargs}}{}
Model for a timeframe object that is common to most financial security charts (e.g. H1, H4, M5)

The naming convention is as follows: the first character of the \emph{name} string is the time unit
(\emph{`M'} = minute, \emph{`H'} = hour, \emph{`D'} = day) and the rest of the string is the number of units for
the timespan.
\index{get\_numeric() (model.models.Period method)}

\begin{fulllineitems}
\phantomsection\label{index:model.models.Period.get_numeric}\pysiglinewithargsret{\bfcode{get\_numeric}}{}{}
Get the number part of the period

\end{fulllineitems}

\index{get\_py\_tdelta() (model.models.Period method)}

\begin{fulllineitems}
\phantomsection\label{index:model.models.Period.get_py_tdelta}\pysiglinewithargsret{\bfcode{get\_py\_tdelta}}{}{}
Get a pythonic \code{timedelta} object for the period implied by the number
\begin{quote}\begin{description}
\item[{Returns}] \leavevmode
the timedelta object, None if the identifier is invalid

\item[{Return type}] \leavevmode
timedelta

\end{description}\end{quote}

\end{fulllineitems}

\index{get\_unit\_letter() (model.models.Period method)}

\begin{fulllineitems}
\phantomsection\label{index:model.models.Period.get_unit_letter}\pysiglinewithargsret{\bfcode{get\_unit\_letter}}{}{}
Get the first letter of the period(resolution) identifier (the unit)

\end{fulllineitems}


\end{fulllineitems}

\phantomsection\label{index:module-model.tests}\index{model.tests (module)}\phantomsection\label{index:module-core.backtest}\index{core.backtest (module)}\index{Tester (class in core.backtest)}

\begin{fulllineitems}
\phantomsection\label{index:core.backtest.Tester}\pysiglinewithargsret{\strong{class }\code{core.backtest.}\bfcode{Tester}}{\emph{configuration}, \emph{start\_dt}, \emph{end\_dt}, \emph{period}}{}
The \code{Tester} class runs a backtest of a given instrument in a given time frame
\index{run() (core.backtest.Tester method)}

\begin{fulllineitems}
\phantomsection\label{index:core.backtest.Tester.run}\pysiglinewithargsret{\bfcode{run}}{}{}
Runs the backtest and provides the results

\begin{notice}{warning}{Warning:}
no drawdown calculation, no support for short-sell
\end{notice}
\begin{quote}\begin{description}
\item[{Return type}] \leavevmode
TesterResult

\end{description}\end{quote}

\end{fulllineitems}


\end{fulllineitems}

\index{TesterConfiguration (class in core.backtest)}

\begin{fulllineitems}
\phantomsection\label{index:core.backtest.TesterConfiguration}\pysiglinewithargsret{\strong{class }\code{core.backtest.}\bfcode{TesterConfiguration}}{\emph{instruments=None}, \emph{weights=None}}{}
Configuration object for the backtest (\emph{Tester})

\end{fulllineitems}

\index{TesterResult (class in core.backtest)}

\begin{fulllineitems}
\phantomsection\label{index:core.backtest.TesterResult}\pysiglinewithargsret{\strong{class }\code{core.backtest.}\bfcode{TesterResult}}{\emph{**kwargs}}{}
Class for encapsulating the backtest results. The class has several attributes:
\begin{quote}

\_return\_pct     The percentage return for the portfolio, over the course of the backtest period
\_return\_nominal The return for the portfolio in nominal terms
\_max\_nominal    The maximum value the portfolio achieves during the timespan
\_min\_nominal    The minimum value the portfolio drops to during the timespan
\end{quote}

\end{fulllineitems}

\phantomsection\label{index:module-core.optimize}\index{core.optimize (module)}
Module includes the optimization related objects
\phantomsection\label{index:module-core.views}\index{core.views (module)}\index{backtest() (in module core.views)}

\begin{fulllineitems}
\phantomsection\label{index:core.views.backtest}\pysiglinewithargsret{\code{core.views.}\bfcode{backtest}}{\emph{req}}{}
View controller for the REST API implementation of the backtest function.

The method is invoked via a POST request to \emph{/backtest/} with the parameters:
\begin{itemize}
\item {} 
``instruments'' : list of instrument names (JSON Array)

\item {} 
``weights''     : list of floating point numbers (JSON Array) for the portfolio weights

\item {} 
``start-date''  : Start date time for the backtest in ISO Datetime format

\item {} 
``end-date''    : End ddate time for the backtest in ISO Datetime format

\item {} 
``period''      : The period specifier (``H1'', ``M5'', etc.)

\end{itemize}

\end{fulllineitems}

\index{optimize() (in module core.views)}

\begin{fulllineitems}
\phantomsection\label{index:core.views.optimize}\pysiglinewithargsret{\code{core.views.}\bfcode{optimize}}{\emph{req}}{}
View controller for the REST API implementation of the portfolio optimization function

\end{fulllineitems}



\chapter{Indices and tables}
\label{index:indices-and-tables}\label{index:pom-api-documentation}\begin{itemize}
\item {} 
\emph{genindex}

\item {} 
\emph{modindex}

\item {} 
\emph{search}

\end{itemize}


\renewcommand{\indexname}{Python Module Index}
\begin{theindex}
\def\bigletter#1{{\Large\sffamily#1}\nopagebreak\vspace{1mm}}
\bigletter{c}
\item {\texttt{core.backtest}}, \pageref{index:module-core.backtest}
\item {\texttt{core.optimize}}, \pageref{index:module-core.optimize}
\item {\texttt{core.views}}, \pageref{index:module-core.views}
\indexspace
\bigletter{m}
\item {\texttt{model.models}}, \pageref{index:module-model.models}
\item {\texttt{model.tests}}, \pageref{index:module-model.tests}
\end{theindex}

\renewcommand{\indexname}{Index}
\printindex
\end{document}
